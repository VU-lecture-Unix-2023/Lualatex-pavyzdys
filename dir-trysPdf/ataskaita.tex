\documentclass{article}
\usepackage{amsmath}
\usepackage{graphicx}
\begin{document}

\author{A.\ Utorius}
\title{Savarankiško darbo Nr.\ 6 atsiskaitymas}
\date{\today}

\maketitle

Daug skaitęs ir sunkiai dirbęs įveikiau visas kliūtis ir sukūriau tris
failus. Jų pirmieji puslapiai yra rodomi sekančiuose puslapiuose
\pageref{fig-linuxOSyes}, \pageref{fig-linuxOSno},
\pageref{fig-testCriteria}.

\clearpage

% ---------------------------
\begin{figure}[p]
\centering
\includegraphics[height=0.9\textheight,width=\textwidth]{pirmas.pdf}
\caption{Linux OS palyginimas.}
\label{fig-linuxOSyes}
\end{figure}
% ---------------------------

\clearpage

% ---------------------------
\begin{figure}[p]
\centering
\includegraphics[height=0.9\textheight,width=\textwidth]{antras.pdf}
\caption{Kada netinka Linux OS.}
\label{fig-linuxOSno}
\end{figure}
% ---------------------------
\clearpage

% ---------------------------
\begin{figure}[p]
\centering
\includegraphics[height=0.9\textheight,width=\textwidth]{trecias.pdf}
\caption{Kaip įvertinti veikiančią Linux OS?}
\label{fig-testCriteria}
\end{figure}
% ---------------------------

\end{document}
