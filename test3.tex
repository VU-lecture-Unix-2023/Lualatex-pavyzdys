\documentclass{article}
\usepackage[utf8x]{inputenc}
\usepackage[T1]{fontenc}
\usepackage[lithuanian]{babel}	% lietuviškas skiemenavimas
\usepackage{amsmath}
\usepackage{graphicx}
\begin{document}

\author{A.\ Utorius}
\title{Lietuvi{\v s}kų raid{\v z}ių ra{\v s}ymas}
\date{\today}

\maketitle

Lietuviškos raid{\.e}s:$<$ĄČĘĖĮŠŲŪŽąčęėįšųūž$>$ \cite{ref1}.
Pavyzdys {\sf test2.tex}, sukompiliuotas su \textsf{pdflatex}.
Lietuviško skiemenavimo paketas buvo įtrauktas.

\section{Apžvalga}

Viena svarbiausių analizinės algebros žinių yra Teiloro eilutė:
\begin{align}
f(x)\big|_{x=a} &=& f(a)
 + \frac{f'(a)}{1!} (x-a)
 + \frac{f''(a)}{2!} (x-a)^2
 + \frac{f'''(a)}{3!} (x-a)^3
 + \dots \nonumber \\
 &=& \sum\limits_{n=0}^{\infty} \frac{f^{(n)}(a)}{n!} (x-a)^n
\end{align}

% ---------------------------
\begin{figure}[b]
\includegraphics[height=0.1\textheight,width=0.1\textwidth]{texlive.png}
\caption{Atlikto darbo pirmasis puslapis.}
\label{fig-report}
\end{figure}
% ---------------------------

% ---------------------------------------------
\section*{Laboratorinio darbo santraukos pavyzdys}

Čia pateikiama trumpa šio darbo santrauka. Joje keliais sakiniais
nurodomas darbo pobūdis, atlikimo būdas, bei gauti rezultatai. Pobūdis
išplaukia iš užduotyje keliamų tikslų (pavyzdžiui, išmatuoti
gravitacijos konstantą $G$), ir turėtų atsakyti į klausimą ,,kas buvo
daroma?{}``. Atlikimo būdas išplaukia iš naudotų matavimo metodų, ir
turėtų atsakyti į klausimą ,,kaip tai buvo daroma?{}``. Gauti
rezultatai pristatomi itin glaustai, pateikiant tik pagrindinius
faktus ar skaitines vertes, atsako į klausimą ,,kas buvo nustatyta,
išmatuota, ar sužinota?{}``. Santraukos apimtis neturėtų būti mažesnė
nei 50 žodžių, ir negali viršyti 100 žodžių.

% ---------------------------------------------
\section*{Padėka}

\textit{\input{greetings}}

\begin{thebibliography}{99}
\bibitem{ref1} A.\ Bėcėlė (unpublished).
\end{thebibliography}
\end{document}
